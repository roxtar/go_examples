\documentclass{beamer}
%\usetheme{Rochester}
\usetheme{Copenhagen}
%\usetheme{Madrid}
\usepackage{graphicx}

\title{The Go Programming Language}
\subtitle{http://www.golang.org \\ CS 239: Parallel Programming Languages}
\author{Raghu Prabhakar \and Rohit Kumar}
\date{June 2, 2011}
\begin{document}

\begin{frame}
\begin{center}
\includegraphics[width=1in]{gopher.png}
\end{center}
\titlepage
\end{frame}

\begin{frame} {About Go}
\begin{itemize}
  \item New language developed by Google. Released in 2009.
  \item Rob Pike, Ken Thompson, Robert Griesemer
  \item Borrows from C, Newsqueak.
  \item Compiled, garbage collected, concurrent, statically typed,
    imperative, native UTF-8 support.
  \item Compilers: 6g, 8g, gcc 
\end{itemize}
\end{frame}

\begin{frame}[fragile]
\frametitle{Concurrency}
  \begin{itemize}
    \item Goroutines
    \item Channels -- send and receive messages
    \item No distributed arrays. 
    \item Others -- WaitGroup, Select, do.Once, etc    
  \end{itemize}
\end{frame}

\begin{frame} [fragile]
\frametitle{Goroutines}
\begin{itemize}
  \item Call \verb=runtime.GOMAXPROCS(n)= to run on multiple processors
  \item \verb=go= statement $\approx$ \verb=async= in X10
\end{itemize}
\begin{center}
\begin{verbatim}
                go mapblock(start, end)
\end{verbatim}
\end{center}
\begin{itemize}
\item Starts function \verb=mapblock= in a new thread with arguments \verb=start= and \verb=end=
\item \verb=mapblock= is called a goroutine
\end{itemize}

\end{frame}

\begin{frame} [fragile]
\frametitle{Channels}
\begin{itemize}
\item Used to send and receive messages between goroutines.
\item Channels are typed
\end{itemize}
\begin{verbatim}
        var ch = make(chan int) 
        ch <- 42 // send 
        var i = <- ch // receive        
\end{verbatim}
\begin{center}
\includegraphics[width=2in]{channel.png}
\end{center}
\end{frame}

\begin{frame}{Channels}

\begin{itemize}
\item Buffered and unbuffered
\item Unbuffered -- block on send and receive
\item Buffered  -- block on send only if buffer is full. Block on receive.

\end{itemize}
\end{frame}

\begin{frame}[fragile]
\frametitle{Channels}
\begin{itemize}
\item Channels can be used to model X10's \verb=finish= statement
\end{itemize}
\begin{verbatim}
        var ch = make(chan bool)
        for i:=0; i < n ; i++ {
                go func() { 
                        // do something
                        ch <- true
                } ()
        }
\end{verbatim}
\pause
\begin{verbatim}
        // drain the channel
        for i:=0; i < n; i++ {
                <- ch
        }
        close(ch)
        
\end{verbatim}
\end{frame}


\begin{frame} [fragile]
\frametitle{WaitGroup}
\begin{itemize}
\item A better construct to model X10's \verb=finish= statement
\end{itemize}
\begin{verbatim}
        var wg sync.WaitGroup
        for i:=0; i < n ; i++ {
                wg.Add(1)
                go func() { 
                        // do something
                        wg.Done()
                } ()                
        }
        wg.Wait() // wait for all go-routines to finish
\end{verbatim}
\end{frame}

\begin{frame} [fragile]
\frametitle{Other Things}
\begin{itemize}
\item The \verb=select= statement
\item sync.Once.Do
\item Locks and mutexes
\item More in the \verb=sync= package
\item \verb=netchan= 
\end{itemize}
\end{frame}


\begin{frame}[fragile] 
\frametitle{Map Reduce}
\begin{itemize}
  \item Sequential
  \item $\frac{n}{k}$ version of map reduce using arrays
  \item Streaming map reduce using channels
\end{itemize}
\end{frame}

\begin{frame} {Parallel Prefix Sum}
\end{frame}

\begin{frame} {Black Scholes}
\end{frame}


\end{document}
