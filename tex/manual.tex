\documentclass[8pt, twocolumn]{article}
\usepackage{hyperref}
\hypersetup{
colorlinks=true,
linkcolor=blue,
citecolor=blue,
urlcolor=blue,
}

\title{Parallel Programming in Go \\
\small{\url{http://www.golang.org}}
} 
\author{
  Raghu Prabhakar
  \\\small{raghu@cs.ucla.edu}
  \and 
  Rohit Kumar 
  \\\small{rohitkk@cs.ucla.edu}
}

\begin{document}

\maketitle
\section{Abstract}
The Go programming language was released by Google in November
2009. The language was designed and implemented by the creators of
Unix: Rob Pike and Ken Thompson. It's a general purpose programming
language with constructs for concurrency. This manual gives a brief
overview of the language's support for concurrency.\\
\section{Introduction}
Go was designed as a general purpose system's programming language. It
borrows from C (pointers, goto), Newsqueak (channels) \cite{gofaq}. It has garbage
collection, is statically typed and has compilers on Windows, Linux
and MacOS X. 
\section{Goroutines}
Goroutines are functions executing in a separate thread. To start a
goroutine, you prefix the function call with the \verb=go= keyword.
\begin{verbatim}
        go map_block(start, end)
\end{verbatim}
The above statement will start the \verb=map_block= function as a
goroutine in a separate thread. This is an asynchronous call and the
control will not wait for the execution of \verb=map_block= to finish
before executing the next command. When the goroutine finishes it will
exit silently. The goroutine shares the same memory
as other goroutines and the main thread of execution. Multiple
goroutines might be multiplexed on the same system thread.\\

Currently by default the Go runtime will only use one processor. To
use more than one processor, one should call the
\verb=runtime.GOMAXPROCS= function. For example:
\begin{verbatim}
        import ("runtime")        
        func main() {
                runtime.GOMAXPROCS(4)
        }
\end{verbatim}
The above code instructs the runtime to use 4 processors.

\section{Channels}
Channels are the main form of syncrhonization provided by the Go
language. They can be used to send and receive values between
goroutines. Channels are typed.
\begin{verbatim}
    1    ch := make(chan int)                         
    2    go func() {
    3            v:= <-ch)
    4    }()
    5    ch <- 23  
\end{verbatim}
On Line 1 we create a new channel using Go's \verb=make=
statement. The default channels are unbuffered and will block on both
send and receive. We then spawn a new goroutine which receives a value
from the channel (Line 3). Finally we send a the number \verb=23=
through the channel (Line 5).\\

To send a value through a channel one uses the \verb=<-= with
the channel on the left hand side (Line 5). To receive a value the
channel is on the right hand side of the \verb=<-= operator. \\

The order of send and receive is important. If we had done
\verb=ch<-23= before Line 2. The program would block and never execute
the \verb=go= statement. This is because an unbuffered channel blocks
both on a send and a receive \cite{effectivego}.

Channels can also be used to wait for goroutines to finish. For
example:
\begin{verbatim}
    1     ch := make(chan bool)
    2     for i:=0; i<n; i++ {
    3             go func() {
    4             //do something
    5                     ch <- true
    6             }()
    7     }
    8     for i:=0; i<n; i++ {
    9             <-ch 
    10    } 
\end{verbatim}
In the above code snippet we spawn an anonymous goroutine \verb=n=
times (Line 3). Each one of these goroutines sends a value \verb=true=
on the channel \verb=ch= after it has finished its task. In the main
goroutine we wait for all child goroutines to finish by receiving the
values on the channels and discarding them (Line 9). We know that
after Line 10, all the spawned goroutines have finished.\\

There are other patterns of using channels, many of
which are illustrated in \cite{effectivego}.
\section{WaitGroup}
Waitgroups are a better construct to syncrhonize the completion of
goroutines. They are present in the \verb=sync= package. The same code
listed in the previous section can be rewritten using a
\verb=WaitGroup=. 
\begin{verbatim}
    1     var wg sync.WaitGroup
    2     for i:=0; i<n; i++ {
    3             wg.Add(1)
    4             go func() {
    5             //do something
    6                     wg.Done()
    7             }()
    8     }
    9     wg.Wait()
    10    } 
\end{verbatim}
Here we see that the main goroutine calls add to set the number of
goroutines to wait for (Line 3). When each goroutine has finished execution, it
calls the \verb=Done= method (Line 6). The main routine waits for the
all the child goroutines to finish by calling \verb=Wait= (Line 9) \cite{sync}.

\section{Select}
The \verb=select= statement is used to choose a send or receive from a
set of channels. The statement is structured like a \verb=switch=
statement, with each case being either a send to or a receive from a
channel. Each of these cases is evaluated from the top to bottom. Out
of all the send/receive expressions which can proceed one is selected
and executed. See \cite{golangspec} for examples.
\section{Locks}
Within the \verb=sync= package, which comes with the standard Go
distribution, there are two types of locks: Mutex and Reader Writer
Lock. The recommended way of synchronizing is channels. The locks
provided here are used to build higher level synchronization
mechanisms. See \cite{sync} for more details.
\section{Once}
The \verb=Once= structure can be used to execute a particular function
only one time. For example:
\begin{verbatim}
    1     var once sync.Once
    2     for i:=0; i<n; i++ {
    3             go func() {
    4                     //do something
    5                     once.Do(cleanup)
    6             }()
    9     }
\end{verbatim}
In the above code although multiple goroutines will reach Line 5, only
one of them will get to execute the \verb=cleanup= function.
\section{Conclusion}
The Go programming language has been around for almost 2 years. Although
it was experimental when it was released it has since been deployed in
production environments \cite{heroku}. Given that it's general
purpose, imperative, easy to learn and provides native concurrency
primitives, we believe that it will be adopted by many more developers in the
future.
\bibliography{manual} 
\bibliographystyle{plain}
\end{document}
